\section{Data Colllection}

Currently, we have not collected any data to analyze for the study due to our setbacks, but we have three different options to get some data to work on:

\textbf{Option 1: Supervisor's Contacts }


The first option is if our supervisor manages to get in contact with the employees at JKPGCITY, and if they are willing to share specific data. This method is likely considered a quantitative approach if the data received by the employees involves structured data (e.g., city statistics) but it could also be a qualitative approach if the data comes through documents or reports that offer deeper insight. 


\textbf{Option 2: Web-Based Data Sets}


The second option involves searching on the web browser to find existing datasets, the dataset could be both qualitative and quantitative resources. 


\textbf{Option 3: Interviews}


Lastly, the third option is conducting interviews, which is a qualitative method. The interviews aim to gain enough information on a certain topic that should be applicable to the algorithms we aim to create.
 

\section{Data Analysis}
Currently, we have two options when it comes to implementing and analyzing data we will collect, we could either use Python or normal HTML, CSS, and JavaScript. The authors are both experienced in both options. 

\textbf{Option 1: Python}


If we were to use Python, the data collected would be stored in MariaDB which is a branch of MySQL. When it comes to database operations, we will use SQLAlchemy or Djangos Object-relational Mapper (ORM). The backend will use Flask or Django to create API endpoints to interact with MariaDB.
Python libraries such as Scikit-learn or TensorFlow will be used to develop prediction algorithms. The backend can process data fetched from MariaDB, apply the prediction model, and send results to the frontend.
When it comes to the feedback mechanism, API endpoints will be implemented to receive user feedback on predictions, which can be used to refine the model.

\textbf{Option 2: HTML, CSS and JavaScript}


The second option will use HTML, CSS, and JavaScript. HTML will be used to design the user interface and then style it with CSS, this includes forms for the user inputs and sections to display predictions.  JavaScript offers different libraries or frameworks such as React or Vue.js, one of them will be used as a framework. It will be used to make the web application interactive, but also to fetch data from the backend using AJAX or Fetch API. Lastly, display predictions, and handle user feedback submissions.


Both options should aim to produce a dynamic application where predictions can be made based on the data stored in the database, and where it will be presented through a web interface and refined through ongoing feedback.

\section{Validity and Reliability}

\section{Considerations}