\section{Data Colllection} \label{sec:DataCollection}

As of now, we have only collected data from the Web Advanced Development which contains the following variables: Name District. We have identified three main methods to collect more data about the city of Jönköping.

\textbf{Option 1: The Internet }


Our first option is to utilize the web browser to collect more data. The data that we find on the internet can be described as both qualitative (Jönköping's districts, educational institutions, or community services) and quantitative (numerical data) resources.


\textbf{Option 2: Surveys}


The second option is to create surveys for both students and teachers. The aim is to gain information about the city which cannot be found on the internet. The answers from the survey can be classed as both quantitative and qualitative. 

\textbf{Option 3: Interviews}


Our third option is conducting interviews, which is a qualitative method. We will interview current and former employees of different stores and buildings in the city of Jönköping.
 

\section{Data Analysis}
Currently, we have two options when it comes to implementing and analyzing data we will collect, we could either use Python or normal HTML, CSS, and JavaScript. The authors are both experienced in both options. 

\textbf{Option 1: Python}


If we were to use Python, the data collected would be stored in MariaDB which is a branch of MySQL. When it comes to database operations, we will use SQLAlchemy or Djangos Object-relational Mapper (ORM). The backend will use Flask or Django to create API endpoints to interact with MariaDB.
Python libraries such as Scikit-learn or TensorFlow will be used to develop prediction algorithms. The backend can process data fetched from MariaDB, apply the prediction model, and send results to the frontend.
When it comes to the feedback mechanism, API endpoints will be implemented to receive user feedback on predictions, which can be used to refine the model.

\textbf{Option 2: HTML, CSS and JavaScript}


The second option will use HTML, CSS, and JavaScript. HTML will be used to design the user interface and then style it with CSS, this includes forms for the user inputs and sections to display predictions.  JavaScript offers different libraries or frameworks such as React or Vue.js, one of them will be used as a framework. It will be used to make the web application interactive, but also to fetch data from the backend using AJAX or Fetch API. Lastly, display predictions, and handle user feedback submissions.


Both options should aim to produce a dynamic application where predictions can be made based on the data stored in the database, and where it will be presented through a web interface and refined through ongoing feedback.

\section{Validity and Reliability}

\section{Considerations}