Decision-making has always been crucial to an organization’s performance and the ability to outperform competitors.  As time has passed, numerous successes and failures have depended on key decisions, creating immense pressure on decision-makers to make smart and prompt choices.

\cite{DECAS} explained that the focus has varied a lot to achieve the best result, and a lot of time throughout the 20th century has been spent on researching subjects such as mathematics, sociology, psychology, economics, and political science. Further, \cite{DECAS} states that researchers have explored how decisions reflect human values and ways to enhance them, resulting in deeper insights into human behavior, improved decision support tools, and ultimately more effective decision-making across different situations.

Studies have been conducted since the mid-20th century where researchers have explored the possibility of decision systems that combined human judgment with machines’ predictive capabilities, triggering interest in trying to improve the decision-making process through technology, data, and analytics. Data-driven decision-making has become more appealing since the rise of data science and machine learning. Combining human intuition with data analysis leads to more logical and successful results, according to \cite{DECAS}.

\cite{DECAS} explains that by combining artificial intelligence (AI) and human intelligence decision-making, many organizations have become motivated to boost their combined intelligence and capabilities, which can result in smarter data analysis and better decision-making support. But the challenging part is to make full use of the data, tools, and insights available without clear guidelines, underling the importance of ongoing research. 

\section{Problem Statement}

With the increase in the field of decision-making by organizations, therefore the integration of data and analytics has become a crucial part of increasing performance and competitive edge. Unfortunately, despite the rich history that exists when it comes to decision theory, and the rise of the importance of big data and analytics, there are still gaps in both research and practice when it comes to a cohesive framework that can combine traditional decision-making elements with modern data-driven insights effectively. The gap becomes clearer when you take into consideration the mixed results that were retrieved from the attempts made by different organizations when it comes to implementing data-driven decision-making strategies. A small percentage of organizations use improvements to make better decisions. While others face problems because they do not fully understand how to use these tools or to combine the elements of humans and machines.

The current decision theories that are used fail to offer guidance on how to integrate big data and advanced analytics with human decision-making processes. This leads to models not fully utilizing the potential of data-driven insights to improve the quality of decisions. With the information provided, it becomes clearer that there is a need for a new theory that recognizes big data and analytics as separate parts and also offers a way to combine it with the human aspect to improve the decision-making process.

The study aims to evaluate DECAS which is a modern data-driven decision theory. DECAS builds on classical decision theory, adding big data and analytics as key elements to the existing framework of decision-making processes, decision-makers, and the decisions themselves.The research aims to investigate how DECAS serves as a complete framework that allows organizations to use the full range of data-driven capabilities to make well-informed, improved, and impactful decisions. In doing this, it aims to contribute to both the theory and practice of computer science, especially in the area of organizational decision-making.

\section{Purpose and Research Questions}
\textbf{Question 1: Can the DECAS decision-making theory apply to visualization analytics for the local dataset from Jönköping?}


With the first question, we aim to see if the DECAS decision-making process could be applied to our collected dataset about Jönköping. This exploration will find out how useful and relevant the theory is in making better sense of and using the data we've collected.


\textbf{Question 2: Whether and how the five pillars (Data, Analytics, Decision Making Process, Decision Maker, and Decision) of DECAS decision-making theory connect to the parts of Tamara Munzner's model (WHAT, WHY, HOW)?}


The second question aims to see whether and how the five pillars of the DECAS decision-making theory are connected to the parts of Tamara Munzner’s model. The comparison aims to identify the synergies and potential improvements for decision-making processes in data visualization.


\textbf{Question 3 (Optional): Whether the outcome of the study could help city officials further the development of the city of Jönköping? }


The outcome of the study will be analyzed, and we will determine whether or not it could be used to help city officials further develop the city of Jönköping. This investigation will analyze the practical applicability of our research findings in real city planning and development work.



\textbf{Question 4 (Optional): Is the decision-making process at JKPGCITY applicable to DECAS decision-making theory or does it need any adjustments?  }


Our supervisor will contact the people listed on the JKPGCITY homepage to arrange an interview session with them which we will discuss their decision-making process. Then with our fourth research question named above, we will try to see if their process applies to the DECAS decision-making process or if it needs adjusting. This means we will talk directly with city officials to learn about how they make decisions now and where they could make things better.




\section{Scope and Limitations}

\subsection{Scope}

\subsection{Limitations}
As written in section \ref{sec:DataCollection}.


\section{Disposition}
The structure of the remainder of the report is as follows: 

\textbf{Chapter 2: Method and Implementation}

This chapter will cover the study´s workflow, including the approach, collection methods, and analysis techniques. It will be divided into sections: Data collection, Data analysis, Validity and reliability and Considerations. The Data Collection section will present the method used to collect the data for the study, while Data Analysis will explain the method chosen to analyze the collected data. The Validity and Reliability section will highlight the importance of ensuring the study’s validity and reliability, explaining the choices made in data collection and analysis. Lastly, the Consideration section will explore important factors such as science, society, the environment, ethics, and sustainability, providing strong justifications for our decisions.




\textbf{Chapter 3: Theoretical Framework} 

This chapter will explore existing research, theories, and resources needed for learning. It will compare and evaluate these theories, forming the basis for creating interview questions, surveys, and analyses.  

\textbf{Chapter 4: Results}

This chapter will display the collected data in a clear and organized way without adding personal opinions. By thoroughly analyzing the data, we will draw conclusions that answer our research questions and meet the study's goals.


\textbf{Chapter 5: Discussion}


This chapter will discuss how our results relate to earlier studies, what our study means, and its limitations. It will include two subchapters: one discussing the results and another discussing the methods in the study.  

 
\textbf{Chapter 6: Conclusions and Further Research}

This final chapter will contain a summary of the key findings of our study and propose directions for further research.  